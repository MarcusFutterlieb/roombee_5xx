\documentclass[a4paper,10pt]{book}
\usepackage[utf8]{inputenc}
\usepackage{minitoc} %to get mini table of contents for each chapter
\usepackage{hyperref} %hyperlinks


\title{Roombee Manual}
\author{Marcus Futterlieb}
\date{20.01.2016}

\pdfinfo{%
  /Title    (roombee manual)
  /Author   (Marcus Futterlieb)
  /Creator  (Marcus Futterlieb) 
  /Producer (Marcus Futterlieb)
  /Subject  (mobile robotics)
  /Keywords (mobile robotics, drive by odometry, diy robot)
}

\begin{document}
\dominitoc
\maketitle

\chapter{Disclaimer}
This software has been developed to allow curious owners of a iRobot Roomba 500 series to gather some experience in the domain of mobile robotics. 
A serious help in the development was "The MATLAB Toolbox for the iRobot Create" which can be found \href{http://www.usna.edu/Users/weapsys/esposito/roomba-matlab.php}{here}.
The author can not be held responsible for any damage caused directly or indirectly by the software. 


\chapter{Prerequisites and Requirements}
\chapter{How to use the Code}
\chapter{How to create the Light Graffiti Effect}
The tool used for post processing the video material is \href{https://kdenlive.org/}{Kdenlive}.
It should be accessible through the apt-get command.
Once installed it handles much like any other video editing software.
The effect you are looking for is called "Light Graffiti" and can be found under:\\
Effect List/Misc/ Light Graffiti

\vspace {0.5cm}

Examples can be found at:

\begin{itemize}
  \item \href{https://youtu.be/hPpPM0kasyk}{Youtube video of a roomba following a certain letter sequence indoors}
  \item \href{https://youtu.be/g-6yrYu4PUs}{Youtube video of a roomba following a certain letter sequence outdoors}
\end{itemize}




\end{document}
